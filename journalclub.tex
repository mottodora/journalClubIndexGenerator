\documentclass[a4j]{jsarticle}
\begin{document}
\title{\vspace{-1.5cm}JournalClub}
\author{Unknown}
\date{2016年5月14日\vspace{-0.3cm}}
\maketitle
\noindent
Nature Volume 533 Number 7602 pp145-284 12 May 2016
\vspace{-5mm}
\section{Articles}
\noindent\textbf{The Atlantic salmon genome provides insights into rediploidizationOpen}
The genome sequence is presented for the Atlantic salmon (Salmo salar), providing information about a rediploidization following a salmonid-specific whole-genome duplication event that resulted in an autotetraploidization.
\vspace{3mm}
\noindent\textbf{Sex-specific pruning of neuronal synapses in Caenorhabditis elegans}
How sex-specific neuronal circuits are generated during development is poorly understood; here, sensory neurons are identified in the round worm Caenorhabditis elegans, which initially connect in both male- and hermaphrodite-specific patterns, but a specific subset of these connections is pruned by each sex upon sexual maturation to produce sex-specific connectivity patterns and dimorphic behaviours.
\vspace{3mm}
\noindent\textbf{Interconnected microbiomes and resistomes in low-income human habitats}
An analysis of bacterial community structure and antibiotic resistance gene content of interconnected human faecal and environmental samples from two low-income communities in Latin America was carried out using a combination of functional metagenomics, 16S sequencing and shotgun sequencing; resistomes across habitats are generally structured along ecological gradients, but key resistance genes can cross these boundaries, and the authors assessed the usefulness of excreta management protocols in the prevention of resistance gene dissemination.
\vspace{3mm}
\section{Letters}
\noindent\textbf{No Sun-like dynamo on the active star ζ Andromedae from starspot asymmetry}
Infrared interferometry imaging of the old, magnetically active star ζ Andromedae reveals an asymmetric distribution of starspots, unlike the north–south starspot symmetry observed on the Sun, meaning the underlying dynamo mechanisms must be different.
\vspace{3mm}
\noindent\textbf{Temperate Earth-sized planets transiting a nearby ultracool dwarf star}
Three Earth-sized planets—receiving similar irradiation to Venus and Earth, and ideally suited for atmospheric study—have been found transiting a nearby ultracool dwarf star that has a mass of only eight per cent of that of the Sun.
\vspace{3mm}
\noindent\textbf{Lightwave-driven quasiparticle collisions on a subcycle timescale}
A quasiparticle collider is developed that uses femtosecond optical pulses to create electron–hole pairs in the layered dichalcogenide tungsten diselenide, and a strong terahertz field to accelerate and collide the electrons with the holes.
\vspace{3mm}
\noindent\textbf{Site-selective and stereoselective functionalization of unactivated C–H bonds}
The idea of carbon–hydrogen functionalization, in which C–H bonds are modified at will, represents a paradigm shift in the standard logic of organic synthesis; here, dirhodium catalysts are used to achieve highly site-selective, diastereoselective and enantioselective C–H functionalization of n-alkanes and terminally substituted n-alkyl compounds.
\vspace{3mm}
\noindent\textbf{Ancient micrometeorites suggestive of an oxygen-rich Archaean upper atmosphere}
Evidence in support of low atmospheric oxygen concentrations on early Earth relates to the composition of the lower Archaean atmosphere; now the composition of fossil micrometeorites preserved in 2.7-billion-year-old rocks in Australia suggests that they were oxidized in an oxygen-rich Archaean upper atmosphere.
\vspace{3mm}
\noindent\textbf{A rapid burst in hotspot motion through the interaction of tectonics and deep mantle flow}
Models of thermochemical convection reveal flow patterns in the deep lower mantle under the north Pacific since 100 million years ago that explain how the enigmatic bend in the Hawaiian–Emperor hotspot track arose.
\vspace{3mm}
\noindent\textbf{First North American fossil monkey and early Miocene tropical biotic interchange}
Here, 21-million-year-old fossils of a New World monkey from Panama are described, constituting the earliest known evidence for mammalian interchange between North and South America.
\vspace{3mm}
\noindent\textbf{Restoring cortical control of functional movement in a human with quadriplegia}
Signals recorded from motor cortex—through an intracortical implant—can be linked in real-time to activation of forearm muscles to restore movement in a paralysed human.
\vspace{3mm}
\noindent\textbf{Self-organization of the in vitro attached human embryo}
An in vitro model to study the early events that direct human embryo development after formation of the blastocyst and implantation in the uterine wall.
\vspace{3mm}
\noindent\textbf{The evolution of cooperation within the gut microbiota}
Little is known about cooperative behaviour among the gut microbiota; here, limited cooperation is demonstrated for Bacteroides thetaiotaomicron, but Bacteroides ovatus is found to extracellularly digest a polysaccharide not for its own use, but to cooperatively feed other species such as Bacteroides vulgatus from which it receives return benefits.
\vspace{3mm}
\noindent\textbf{Molecular mechanism of APC/C activation by mitotic phosphorylation}
Phosphorylation of the anaphase-promoting complex (APC/C) allows for its control by the co-activator Cdc20; a mechanism that has relevance to understanding the control of other large multimeric complexes by phosphorylation.
\vspace{3mm}
\noindent\textbf{Activation of the A2A adenosine G-protein-coupled receptor by conformational selection}
The adenosine A2A receptor, a class A G-protein-coupled receptor, exists as an ensemble of two inactive and two active states in equilibrium and is activated by conformational selection rather than induced fit.
\vspace{3mm}
\noindent\textbf{Architecture of the mitochondrial calcium uniporter}
The structure of the core region of the mitochondrial calcium uniporter (MCU) is determined by NMR and electron microscopy, revealing that MCU is a homo-pentamer with a specific transmembrane helix forming a hydrophilic pore across the membrane, and representing one of the largest membrane protein structures characterized by NMR spectroscopy.
\vspace{3mm}
\noindent\textbf{Extra-helical binding site of a glucagon receptor antagonist}
The X-ray crystal structure of the transmembrane portion of the human glucagon receptor, a class B G-protein-coupled receptor (GPCR), is solved in the presence of the antagonist MK-0893, with potential implications for the development of therapeutics that target other class B GPCRs.
\vspace{3mm}
\end{document}
